\documentclass{article}

% Language setting
% Replace `english' with e.g. `spanish' to change the document language
\usepackage[english]{babel}

% Set page size and margins
% Replace `letterpaper' with`a4paper' for UK/EU standard size
\usepackage[letterpaper,top=2cm,bottom=2cm,left=3cm,right=3cm,marginparwidth=1.75cm]{geometry}

% Useful packages
\usepackage{amsmath}
\usepackage{graphicx}
\usepackage[colorlinks=true, allcolors=blue]{hyperref}

\title{Your Paper}
\author{You}

\begin{document}

\section{Exercice 5}
\subsection{}
Soit le système suivant :
$$
\left\{
    \begin{array}{ll}
        x'(t)=-y(t) + u(t)cos(t) \\
        y'(t)=x(t)+u(t)sin(t)
    \end{array}
\right.
$$
Linéarisons le système, il peut donc s'écrire de la forme suivante :
\[
\begin{pmatrix}
x'\\
y'
\end{pmatrix} =\begin{pmatrix}
0 & -1 \\
1 & 0 
\end{pmatrix}\begin{pmatrix}
x\\
y 
\end{pmatrix} + \begin{pmatrix}
cos(t)\\
sin(t)
\end{pmatrix} u(t) = A \begin{pmatrix}
x \\
y 
\end{pmatrix} + B u(t)
\]
avec $A =\begin{pmatrix}
0 & -1 \\
1 & 0 
\end{pmatrix}$ et $B = \begin{pmatrix}
cos(t) \\
sin(t)
\end{pmatrix}$
\newline{}
On a donc n=2 et m=1, la matrice Kal(A,b) sera de la forme : 
\[
Kal(A,B)=[B|AB]=\begin{pmatrix}
cos(t) & -sin(t)  \\
sin(t) & cos(t)
\end{pmatrix}
\]
donc :
\[
det(Kal(A,B))= 1 \neq 0 \quad \forall t \in [0,T]
\]
On a donc que le rang de la matrice est maximale. le système est linéaire, mais pas autonome (pas de la forme x’ = Ax+Bu avec A et B ne dépendant pas de t). On a donc que le  critère de Kalman ne dit rien sur la contrôlabilité dans ce cas.

\subsection{}
Posons $g(t)=x(t)sin(t) - y(t)cos(t)$ \newline{}
alors : 
\[ g'(t)=x'(t)sin(t) + x(t)cos(t) - y'(t)cos(t) + y(t)sin(t)
\]
or :
$$
\left\{
    \begin{array}{ll}
        x'(t)=-y(t) + u(t)cos(t) \\
        y'(t)=x(t)+u(t)sin(t)
    \end{array}
\right.
$$
donc :
\[g'(t)=-y(t)sin(t)+u(t)cos(t)sin(t)+x(t)cos(t) - x(t)cos(t) -u(t)cos(t)sin(t) + y(t)sin(t)\]
et donc :  \[ g'(t)=0
\]
donc g(t) est constante quelque soit t. \newline{}
En particulier $g(T)=g(0)=c_1$ avec $c_1$ une constante.\newline{}
et ainsi : 
\[
(*)\quad  x(T)sin(T) - y(T)cos(T)=-y(0)
\]
en posant T=2$\pi$, (*) se réécrit y(2$\pi$)=y(0)\newline{}
Or, posont maintenant en temps T = 2$\pi$:
\[
\begin{pmatrix}
        x \\
        y
    \end{pmatrix}=\begin{pmatrix}
        0 \\
        y(0)+1
    \end{pmatrix}
\]
Mais y($2\pi$)=y(0) donc y(2$\pi$) $\neq$ y(0) + 1, donc on ne peut pas trouver u tel que y(2$\pi$) = y(0)+1 donc le système n'est pas contrôlable en temps $2\pi$.
\end{document}
